\chapter{Introduction}
\pagenumbering{arabic}
Do look up report style and requirements of your department and ensure that the report follows the standard set by your department. For example, the page margins can be set under the package \textit{geometry}. It is advisable to look up the margin requirements of your department. 

If you are not familiar with concepts like styles, captioning, cross-referencing, and how to generate tables of contents, figures etc. in LaTeX, the Overleaf guides are a useful start at: \url{https://www.overleaf.com/learn/latex/Learn_LaTeX_in_30_minutes}. 

Otherwise, a lot of help can be obtained through Stack Exchange, a forum where people get help from others regarding \LaTeX\ writing. 

Other useful tools are
\begin{itemize}
    \item Detexify, which can convert your drawings to \LaTeX\ commands, 
    \item Mathpix tools, which you can use to screenshot an equation and it will convert into \LaTeX\ equation form to be copied into your .tex document,
    \item Grammarly is a great help for grammar assistance on overleaf. 
\end{itemize}

\section{Sectioning}
This is a section. You can also use subsections but subsubsections should be avoided if possible. If used, it should be unnumbered using \textbackslash subsubsection*\{\}. 
\subsection{Subsection}
Example of a subsection. 

\subsubsection*{Subsubsection}
Example of a sub-subsection. \\
\\This document will have a brief reminder of report writing conventions (IEEE style). Then, some templates will be included, which can be copied from the editor for ease of use. The rest of the document is irrelevant to the template. 