\chapter{Templates of \LaTeX\ groups}
List of groups which can later be copied and pasted to keep the style while changing the content of the groups. 

\section{Equations}
Equation \ref{Emc2_symb} is the famous Einstein mass-energy equation:
\begin{equation}\label{Emc2_symb}
    E = mc^2,
\end{equation}
where $E$ is energy, $m$ is mass and $c$ represents the speed of light. This can also be written in text \textendash\ \(E = mc^2\). The terms must be explained, and equations labelled. 

\section{Figures and subfigures}
Subfigures require the package \textit{subfigure}. Figure \ref{fig:TCD} is a typical figure and Figure \ref{fig:subfig} is in a subfigure style. The captions should be elaborate and clearly describe the figure. Use \[\] in the figure captions if you wish to shorten the figure details at the list of figures. 

\begin{figure}[htbp!]
\centering
\includegraphics[width=0.6\linewidth]{0_title/Trinity_Main_Logo.jpg}
\caption[Trinity College Dublin logo.]{The logo of Trinity College Dublin. Image taken from the Trinity College Dublin website \cite{TCD_Logo}.}
\label{fig:TCD}
\end{figure}

\begin{figure}[!htbp]
        \hfill
	\centering
	\subfigure[Caption 1]{
		\label{fig:subfig1}
        \includegraphics[width=0.45\linewidth]{0_title/Trinity_Main_Logo.jpg}}
        \hfill
	\subfigure[Caption 2]{
		\label{fig:subfig2}
		\includegraphics[width=0.45\linewidth]{0_title/Trinity_Main_Logo.jpg}}
        \hfill
	\caption[Short caption for list of figures.]{The logo of Trinity College Dublin. Image taken from the Trinity College Dublin website \cite{TCD_Logo}.}
	\label{fig:subfig}
\end{figure}

\section{Simple and complex tables}
Table \ref{tab:example} is a dummy table. Table \ref{tab:multi} is a multi-row and multi-column table, which needs the package \textit{multicol}. 

\begin{table}[!h]
    \centering
    \caption{Each row should be a different manipulated variable while each column is the corresponding results obtained.}
    \begin{tabular}{r|c c}
        Manipulating var. / \textit{unit} & Responding var. 1 / \textit{unit} & Responding var. 2 / \textit{unit} \\
        \hline
        Option 1 & Outcome 1-1 & Outcome 2-1 \\
        Option 2 & Outcome 2-1 & Outcome 2-2 \\
    \end{tabular}
    \label{tab:example}
\end{table}

\begin{table}[!h]
  \caption{A complex table with multirows and multicolumns.}
  \centering
    \begin{tabular}{c|c c c c}
    \hline 
    \multirow{2}{*}{Year} & Quantity 1 & Quantity 2 & \multicolumn{2}{c}{Quantity 3} \\
        & (a.u.) & (m) & 3.1 (\%) & 3.2 (\%) \\ 
    \hline 
    2007 & $5\times5\times0.3$ & x & 10 & 90 \\
    2008 & $5\times5\times0.3$ & y & 10 & 80 \\
    2009 & $12\times12\times0.3$ & z & 20 & 70\\[0.8ex]
    \hline
    \end{tabular}
  \label{tab:LSCConfig}
\end{table}

\begin{table}[!htbp]
\caption{Using \textit{threeparttable} to add footnotes to a table (not recommended).}
  \begin{center}
  \begin{threeparttable}
    \begin{tabular}{c c c}
    \hline
    Letter & Number & Description \\
    \hline
    A & 1 & \multirow{3}{*}{\(\ast\)} \\
    B & 2 & \\
    C & 3 & \\
    D & 10 & No same pattern \\
    \hline
    \end{tabular}
    \begin{tablenotes}[para,flushleft]
    \(\ast\) This is a footnote for the table. 
    \end{tablenotes}
  \end{threeparttable}
  \end{center}
  \label{tab:dye}
\end{table}

\break
\section{Bullet points}
Using \textit{enumerate} for numbered points while \textit{itemize} for bullet points. The points should be written in a sentence. For example, 
\begin{enumerate} \label{pt:num}
    \item point 1,
    \item point 2, 
\end{enumerate}
\begin{itemize} \label{pt:item}
    \item item 1, 
    \item item 2. 
\end{itemize}
If the items can be in paragraphs without using bullet points, that will be much better. 

\section{Nomenclature}
The speed of light is \textit{c}, while PV is an abbreviation for photovoltaic.
\nomenclature[P]{\(c\)}{\href{https://physics.nist.gov/cgi-bin/cuu/Value?c}{The speed of light in vacuum}\nomunit{\SI{299792458}{\meter\per\second}}}
\nomenclature[A]{PV}{Photovoltaic}