\chapter{Evaluation}
\subsection{Scientific Analysis}
When working on analysis, making assumptions is good if insufficient information is provided. However, any assumptions made should be clearly stated and described, and if possible, elaborated such that the assumptions made are logical. This also makes the assumptions thought out and can be scrutinised and analysed. It gives the readers some understanding of potential errors and uncertainties that arise from any of the data presented in the report, such that if any replication of research was done, potential deviation from the presented data in the research can be easily determined. 

When presenting your values, take note of the use of significant figures or decimal places. The main thing that is driving the accuracy of the data is the uncertainty or error of the reading. Generally, high accuracy (thus a large number of significant figures) data is unlikely, and the general rule is to have the same number of significant figures as the value of the lowest significant figures in the calculation. However, if a value has a known uncertainty, it can be written until that level of accuracy (e.g. a measuring tape has an accuracy of up to 1mm, so any measurements of length can have up to 3 decimal points in m). 

Besides, when presenting the observation of data, any interesting findings should be discussed and analysed. If no interesting data is found, it is fine to just state the findings are trivial or within expectation and move on. Any data that is presented in the report should be discussed. Any data that is not discussed should not be presented in the report. If there is weird data that is suspicious, it is fine to just say that the results obtained are not within expectations, and discuss the possible errors causing the unexpected results, or potentially discuss what can be improved to obtain better results. 

If an unexplainable result is obtained, it is fine to include it in the report but give some potential future work that can be done to resolve the error or explore the new discovery. Scientific work is a process of learning, and it is fine to obtain results that cannot be clearly explained. It is better to acknowledge the lack of understanding than to brush it off. 