\chapter{Some helpful reminders}
\section{Citations and Reference}
The common two ways of doing it is by numbers (generally Vancouver style \cite{Imper_VancouverGuide, Imper_VancouverGuideLaTeX}) or by author-year (generally Harvard style \cite{Imper_HarvGuide, Imper_HarvGuideLaTeX}). These citations are clickable and will bring you to a general guide on how to prepare your references in general and also in \LaTeX, prepared by the Imperial College London library. 

If paraphrasing can be done, that would be much more recommended than a direct quote. However, if the message cannot be transmitted through paraphrasing, a limited amount of direct quoting can be done. Also, self-plagiarism is still considered plagiarism!

\section{Typography} \label{f:typography}
Typically, scientific writing should be in the passive voice. The use of active voice has been more and more accepted in scientific writing, however, it is generally better to be in a passive voice. The \textbf{bold} style is not recommended to be used for emphasis. Excessive (or any) use of \textbf{bold} in text is frowned upon. Emphasis is usually done using \emph{italics}. 

The open quotation mark (`) is different to the close quotation mark ('). An easy way to avoid this issue is to use the command \textit{enquote}, which needs the package \textit{csquotes}. 

By convention, \textit{italics} should only be used for mathematical terms (e.g. \textit{l} for length, \textit{m} for mass), while units should be in normal style (e.g. m for metre, and kg for kilogram). When writing symbols, it is generally recommended to keep one symbol to represent one quantity, while using subscripts to specify the specific quantity it represents. For example, for the population of Dublin, $P_{D}$ is good while $Population_{Dublin}$ or $Pop_{Dublin}$ are not recommended. Always explain any variables introduced in the report. 